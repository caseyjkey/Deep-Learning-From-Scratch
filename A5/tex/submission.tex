% This contents of this file will be inserted into the _Solutions version of the
% output tex document.  Here's an example:

% If assignment with subquestion (1.a) requires a written response, you will
% find the following flag within this document: <SCPD_SUBMISSION_TAG>_1a
% In this example, you would insert the LaTeX for your solution to (1.a) between
% the <SCPD_SUBMISSION_TAG>_1a flags.  If you also constrain your answer between the
% START_CODE_HERE and END_CODE_HERE flags, your LaTeX will be styled as a
% solution within the final document.

% Please do not use the '<SCPD_SUBMISSION_TAG>' character anywhere within your code.  As expected,
% that will confuse the regular expressions we use to identify your solution.

\def\assignmentnum{5 }
\def\assignmentname{Course Scheduling}
\def\assignmenttitle{XCS221 Assignment \assignmentnum --- \assignmentname}
\input{macros}
\begin{document}
\pagestyle{myheadings} \markboth{}{\assignmenttitle}

% <SCPD_SUBMISSION_TAG>_entire_submission

This handout includes space for every question that requires a written response.
Please feel free to use it to handwrite your solutions (legibly, please).  If
you choose to typeset your solutions, the |README.md| for this assignment includes
instructions to regenerate this handout with your typeset \LaTeX{} solutions.
\ruleskip

\LARGE
0.a
\normalsize

% <SCPD_SUBMISSION_TAG>_0a
\begin{answer}
  % ### START CODE HERE ###
For my CSP, I define a variable for each switch $X_1, \ldots, X_m$, where $X_i$ is the state of switch $i$. The domain of $X_i$ is $\{0, 1\}$ where $0 =$ off and $1 =$ on.

\textbf{Constraints:}
\begin{itemize}
    \item We create one constraint for each light bulb $j$, where $j$ ranges from $1$ to $n$
    \item Each light bulb starts as off ($0$)
    \item A light is on if it has been toggled an odd number of times
    \item Each lightbulb is an element of $T_j$, and it is controlled by a set of switches $X_j$
    \item Scope: $\{X_i \mid i \in T_j\}$
    \item Expression: $\left(\sum_{\{i : i \in T_j\}} X_i\right) \bmod 2 == 1$
\end{itemize}
  % ### END CODE HERE ###
\end{answer}
% <SCPD_SUBMISSION_TAG>_0a
\clearpage

\LARGE
0.b
\normalsize

% <SCPD_SUBMISSION_TAG>_0b
\begin{answer}
  % ### START CODE HERE ###
\textbf{i.} Two satisfying assignments: $(0,1,0)$ and $(1,0,1)$

\textbf{ii.} Backtracking without forward checking:
\begin{enumerate}
    \item backtrack(\{\})
    \item backtrack(\{$X_1: 0$\})
    \item backtrack(\{$X_1: 0, X_3: 0$\}) \\
    try $X_2$:
    \begin{itemize}
        \item try 0: $X_1 \oplus X_2 = 0$ (conflict)
        \item try 1: $X_1 \oplus X_2 = 1$, $X_2 \oplus X_3 = 1$ \\
        backtrack(\{$X_1: 0, X_3: 0, X_2: 1$\}) - solution: $(0, 1, 0)$
    \end{itemize}
    \item backtrack(\{$X_1: 0, X_3: 1$\}) \\
    try $X_2$:
    \begin{itemize}
        \item try 0: $X_1 \oplus X_2 = 0$
        \item try 1: $X_1 \oplus X_2 = 1$, $X_2 \oplus X_3 = 0$
    \end{itemize}
    \item backtrack(\{$X_1: 1$\})
    \item backtrack(\{$X_1: 1, X_3: 0$\}) \\
    try $X_2$:
    \begin{itemize}
        \item try 0: $X_1 \oplus X_2 = 1$
        \item try 1: $X_1 \oplus X_2 = 0$, $X_2 \oplus X_3 = 0$
    \end{itemize}
    \item backtrack(\{$X_1: 1, X_3: 1$\}) \\
    try $X_2$:
    \begin{itemize}
        \item try 0: $X_1 \oplus X_2 = 1$, $X_2 \oplus X_3 = 1$ \\
        backtrack(\{$X_1: 1, X_3: 1, X_2: 0$\}) - solution: $(1, 0, 1)$
    \end{itemize}
\end{enumerate}

\textbf{iii.} Backtracking with forward checking:
\begin{enumerate}
    \item backtrack(\{\})
    \item backtrack(\{$X_1: 0$\}) \\
    $X_2 \to \{1\}$
    \item backtrack(\{$X_1: 0, X_3: 0$\}) \\
    $X_2 \to \{1\}$ \\
    backtrack(\{$X_1: 0, X_3: 0, X_2: 1$\}) - solution: $(0, 1, 0)$
    \item backtrack(\{$X_1: 1$\}) \\
    $X_2 \to \{0\}$
    \item backtrack(\{$X_1: 1, X_3: 1$\}) \\
    $X_2 \to \{0\}$ \\
    backtrack(\{$X_1: 1, X_3: 1, X_2: 0$\}) - solution: $(1, 0, 1)$
\end{enumerate}
  % ### END CODE HERE ###
\end{answer}
% <SCPD_SUBMISSION_TAG>_0b
\clearpage

\LARGE
2.d
\normalsize

% <SCPD_SUBMISSION_TAG>_2d
\begin{answer}
  % ### START CODE HERE ###
Yes, my profile.txt generates a reasonable schedule. It abides by the max units counts and assigns a course for both quarters using my preferred courses.

Profile.txt
minUnits 3
maxUnits 10

register Aut2026
register Win2027

request CS106A
request CS105
request CS101

Schedule

Quarter         Units   Course

  Aut2026       5       CS105
  
  Win2027       5       CS106A
  % ### END CODE HERE ###
\end{answer}
% <SCPD_SUBMISSION_TAG>_2d
\clearpage

\LARGE
3.a
\normalsize

% <SCPD_SUBMISSION_TAG>_3a
\begin{answer}
  % ### START CODE HERE ###
I would change factor A to increase the number of total residents. The constraints are in play to ensure the health and well-being of the residents. Without them, they could become overworked and that would lead to medical errors that would harm patients. It is best to prioritize safety standards.
  % ### END CODE HERE ###
\end{answer}
% <SCPD_SUBMISSION_TAG>_3a
\clearpage

% <SCPD_SUBMISSION_TAG>_entire_submission

\end{document}
