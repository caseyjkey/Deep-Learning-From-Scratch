\item {\bf Safe Exploration}

We learned about different state exploration policies for RL in order to get information about \texttt{(s,a)}. 
The method implemented in our MDP code is epsilon-greedy exploration, which balances both exploitation (choosing the action a that 
maximizes $\hat{Q}_{\text{opt}}(s, a))$ and exploration (choosing the action a randomly).
$$\pi_{\text{act}}(s) = \begin{cases} \arg\max_{a \in \text{Actions}}\hat{Q}_{\text{opt}}(s,a) & \text{probability } 1 - \epsilon \\ 
\text{random from Actions}(s) & \text{probability } \epsilon \end{cases}$$

In real-life scenarios when safety is a concern, there might be constraints that need to be set in the state exploration phase. 
For example, robotic systems that interact with humans should not cause harm to humans during state exploration. 
\cite{fn-2} in RL is thus a critical research question in the field of AI safety 
and human-AI interaction. 

Assume there are harmful consequences for the driver of a mountain-car if the car exceeds a certain velocity. 
One very simple approach of constrained RL is to restrict the set of potential actions that the agent can take at each step. 
We want to apply this approach to restrict the states that the agent can explore in order to prevent reaching unsafe speeds.

\begin{enumerate}

  \input{05-safe-exploration/01-max-speed}

  \item \points{4b}

For Q-learning in continuous states, we need to use function approximation. The first step of function approximation is extracting 
features given state. Feature extractors of different complexities work well with different problems: linear and polynomial feature 
extractors that work well with simpler problems may not be suitable for other problems. For the mountain car task, we are going to use 
a Fourier feature extractor. As background, any continuous periodic function can be approximated as a Fourier Series
$$f(x) = \frac{a_0}{2} + \sum_{j=1}^n\left[a_j\cos(2\pi j x/T) + b_j\sin(2\pi j x/ T)\right]$$
with $a_j$ and $b_j$ sequences of coefficients determined by integrating $f$. To apply this to Q-learning with function approximation,
we want the learned weights $w$ to emulate $a_j$ and $b_j$ and the output of $\phi$ to provide the basis of varying sinusoid periods
as seen in $\cos(2\pi j x/T)$ for $j = 1, 2, \ldots, n$.
Thus, for state $s = [s_1, s_2, \ldots, s_k]$, action $a$, and maximum coefficient $c$, the feature extractor $\phi$ is:
$$\phi(s, a, c) = [\cos(0), \cos(s_1), \ldots, \cos(s_k), \cos(2s_1), \cos(s_1+s_2), \ldots, \cos(cs_1 + cs_2 + \ldots + cs_k)]$$
$$= \left\{\cos\left(\sum_{i=1}^k c_i s_i\right): \forall c_1\in\mathscr{C}, \ldots, c_k\in\mathscr{C}\right\}$$
where $\mathscr{C} = \{0, 1, \ldots, c\}$ is the set of non-negative integers from 0 to $c$. Note that $\phi(s, a, c) \in \mathbb{R}^{(c+1)^k}$.

Implement |fourierFeatureExtractor| in |submission.py|. Looking at |util.polynomialFeatureExtractor| may be useful
for familiarizing oneself with numpy.




  \input{05-safe-exploration/03-constrained}

  \input{05-safe-exploration/04-real-world-context}

\end{enumerate}
