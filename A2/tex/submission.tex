% This contents of this file will be inserted into the _Solutions version of the
% output tex document.  Here's an example:

% If assignment with subquestion (1.a) requires a written response, you will
% find the following flag within this document: <SCPD_SUBMISSION_TAG>_1a
% In this example, you would insert the LaTeX for your solution to (1.a) between
% the <SCPD_SUBMISSION_TAG>_1a flags.  If you also constrain your answer between the
% START_CODE_HERE and END_CODE_HERE flags, your LaTeX will be styled as a
% solution within the final document.

% Please do not use the '<SCPD_SUBMISSION_TAG>' character anywhere within your code.  As expected,
% that will confuse the regular expressions we use to identify your solution.

\def\assignmentnum{2 }
\def\assignmentname{Route Planning}
\def\assignmenttitle{XCS221 Assignment \assignmentnum --- \assignmentname}
\input{macros}
\begin{document}
\pagestyle{myheadings} \markboth{}{\assignmenttitle}

% <SCPD_SUBMISSION_TAG>_entire_submission

This handout includes space for every question that requires a written response.
Please feel free to use it to handwrite your solutions (legibly, please).  If
you choose to typeset your solutions, the |README.md| for this assignment includes
instructions to regenerate this handout with your typeset \LaTeX{} solutions.
\ruleskip

\LARGE
1.a
\normalsize

% <SCPD_SUBMISSION_TAG>_1a
\begin{answer}

    The minimum cost to reach $(m,n)$ from $(0,0)$ is
    \[
        (m+n) + \frac{m(m-1)}{2}.
    \]
    One minimum--cost path is to move $n$ steps up, then $m$ steps right.  
    This is unique because the upward steps are always cheaper and must be taken first before moving right.



\end{answer}
% <SCPD_SUBMISSION_TAG>_1a
\clearpage

\LARGE
1.b
\normalsize

% <SCPD_SUBMISSION_TAG>_1b
\begin{answer}
    \begin{enumerate}
        \item[i)] False, because moving right will always eventually be cheaper than continuing upwards.
        \item[ii)] False, since moving upwards is cheaper than moving right, so $y$ will exceed $n$ before reaching $m$.
        \item[iii)] True, because UCS uses a priority queue to explore only the cheapest actions first.
    \end{enumerate}

\end{answer}
% <SCPD_SUBMISSION_TAG>_1b
\clearpage

\LARGE
1.c
\normalsize

% <SCPD_SUBMISSION_TAG>_1c
\begin{answer}
  % ### START CODE HERE ###
      \begin{enumerate}
        \item[i)] True, since the minimum--distance path cost will remain the same or decrease if the new edge is cheaper.
        \item[ii)] True, because the changed edge might become negative enough to offset the surrounding path and become the new minimum.
        \item[iii)] False, since a longer path with zero--cost actions could become more expensive than a shorter path.
    \end{enumerate}
  % ### END CODE HERE ###
\end{answer}
% <SCPD_SUBMISSION_TAG>_1c
\clearpage

\LARGE
2.b
\normalsize

% <SCPD_SUBMISSION_TAG>_2b
\begin{answer}
  % ### START CODE HERE ###
  The path from AOERC to the Oval assumes no foot traffic, but it goes through busy areas such as buildings that are likely to be congested.  
    Our model does not account for this. I learned this is the closest parking to the Oval, which surprised me because there is parking left of the Oval, but it is not part of the shortest path.
  % ### END CODE HERE ###
\end{answer}
% <SCPD_SUBMISSION_TAG>_2b
\clearpage

\LARGE
3.b
\normalsize

% <SCPD_SUBMISSION_TAG>_3b
\begin{answer}
  % ### START CODE HERE ###
   The maximum number of states visited for $n$ locations with $k$ waypoints is
    \[
        n \cdot 2^k.
    \]
    We can visit each location with $2^k$ different possibilities for which tags have been collected.

  % ### END CODE HERE ###
\end{answer}
% <SCPD_SUBMISSION_TAG>_3b
\clearpage

\LARGE
3.c
\normalsize

% <SCPD_SUBMISSION_TAG>_3c
\begin{answer}
  % ### START CODE HERE ###
  I made a route from the Oval to AOERC with waypoints for food and parking.  
    I learned that my endpoint is the best place to park if I plan to get food before going to AOERC after starting my day at the Oval.  
    I might consider bringing a skateboard since this appears to be a long route.  
    I expected it to be long because the Oval and AOERC are on opposite sides of campus.
  % ### END CODE HERE ###
\end{answer}
% <SCPD_SUBMISSION_TAG>_3c
\clearpage

\LARGE
4.d
\normalsize

% <SCPD_SUBMISSION_TAG>_4d
\begin{answer}
  % ### START CODE HERE ###
  To have the same time complexity for A* with the no-waypoints heuristic and the relaxed shortest--path problem,  
    the waypoints must lie along the diagonal toward $(9,9)$.  
    For example: $(1,1), (2,2), \ldots, (9,9)$.
  % ### END CODE HERE ###
\end{answer}
% <SCPD_SUBMISSION_TAG>_4d
\clearpage

% <SCPD_SUBMISSION_TAG>_entire_submission

\end{document}