\item \points{4c}

Now we can find the Q-value of each |(state, action)| by multiplying the extracted features from this pair with weights. 
Unlike the tabular Q-learning in which Q-values are updated directly, with function approximation, we are updating weights associated 
with each feature. Using |fourierFeatureExtractor| from the previous part, complete the implementation of |FunctionApproxQLearning|.
