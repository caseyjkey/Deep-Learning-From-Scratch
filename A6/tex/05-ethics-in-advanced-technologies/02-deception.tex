\item \points{5b}

Self-driving cars and tracking technologies can potentially be hacked by bad actors to purposefully cause unsafe behavior 
(e.g., to harm people within a specific community). Research on adversarial ML focuses on how attacks like these can be carried out 
against machine learning systems and how to develop safeguards against such attacks.
See \href{https://www.technologyreview.com/2017/08/22/242124/hackers-are-the-real-obstacle-for-self-driving-vehicles/}{https://www.technologyreview.com/2017/08/22/242124/hackers-are-the-real-obstacle-for-self-driving-vehicles/}.
    
Two specific examples of adversarial ML attacks on self-driving cars include: 

\begin{itemize}
    \item "Cloaking and stealthy attacks" to hide the presence of another vehicle from being detected
    \item Sensor deception to make it seem like vehicles are closer or farther than they actually are
\end{itemize}
    
Assume that sensor deception affects the problem setup from Problem 2 as follows. 
The observed distance $D_{t}^{'}$ after attack (see below) is now a skewed Gaussian distribution. 
To apply skewness, apply the adjustment below to the observed distance variable in 
|submission.py| |ExactInferenceWithSensorDeception.observe()| (we are using a technique called Box-Muller transform 
but you don’t need to know the details of this method for this problem). The skewness factor is set in the 
|__init__| constructor with a default value of 0.5 (and you can assume the factor will always be greater than 0). 
Implement this change in the code and rerun |python drive.py -a -p -d -k 3 -i exactInferenceWithSensorDeception| to observe any changes. 

$D_{t}^{'} = \frac{1}{(1+\text{skewness}^2)} * D_t + \sqrt{2 * \frac{1}{(1+\text{skewness}^2)}}$

\textbf{What we expect:} Update the code in |submission.py| |observe()| for class |ExactInferenceWithSensorDeception|. 

